%!TEX TS-program = xelatex
%!TEX encoding = UTF-8 Unicode
% Awesome CV LaTeX Template for CV/Resume
%
% This template has been downloaded from:
% https://github.com/posquit0/Awesome-CV
%
% Author:
% Claud D. Park <posquit0.bj@gmail.com>
% http://www.posquit0.com
%
% Modifier:
% Ben Liu <ben196888@gmail.com>
% https://benliu.me
%
% Template license:
% CC BY-SA 4.0 (https://creativecommons.org/licenses/by-sa/4.0/)
%


%-------------------------------------------------------------------------------
% CONFIGURATIONS
%-------------------------------------------------------------------------------
% A4 paper size by default, use 'letterpaper' for US letter
\documentclass[11pt, a4paper]{awesome-cv}

% Configure page margins with geometry
\geometry{left=1.4cm, top=.8cm, right=1.4cm, bottom=1.8cm, footskip=.5cm}

% Specify the location of the included fonts
\fontdir[fonts/]

% Color for highlights
% Awesome Colors: awesome-emerald, awesome-skyblue, awesome-red, awesome-pink, awesome-orange
%                 awesome-nephritis, awesome-concrete, awesome-darknight
\colorlet{awesome}{awesome-darknight}
% Uncomment if you would like to specify your own color
% \definecolor{awesome}{HTML}{CA63A8}

% Colors for text
% Uncomment if you would like to specify your own color
% \definecolor{darktext}{HTML}{414141}
% \definecolor{text}{HTML}{333333}
% \definecolor{graytext}{HTML}{5D5D5D}
% \definecolor{lighttext}{HTML}{999999}

% Set false if you don't want to emphasise firstname
\setbool{acvHeaderFirstnameFontLight}{false}

% Set false if you don't want to highlight firstname with grey color
\setbool{acvHeaderFirstnameColorGrey}{false}

% Set false if you don't want to highlight section with awesome color
\setbool{acvSectionColorHighlight}{false}


% If you would like to change the social information separator from a pipe (|) to something else
\renewcommand{\acvHeaderSocialSep}{\quad\textbar\quad}


%-------------------------------------------------------------------------------
%	PERSONAL INFORMATION
%	Comment any of the lines below if they are not required
%-------------------------------------------------------------------------------
% Available options: circle|rectangle,edge/noedge,left/right
% \photo[rectangle,edge,right]{./examples/profile}
\name{Ben}{Liu}
\position{Senior Software Engineer}
% \address{}

\mobile{(+66) 825-811373}
\email{ben196888@gmail.com}
% \homepage{benliu.me}
\github{ben196888}
\linkedin{ben-liu}
% \gitlab{gitlab-id}
% \stackoverflow{SO-id}{SO-name}
% \twitter{@twit}
% \skype{skype-id}
% \reddit{reddit-id}
% \medium{madium-id}
% \googlescholar{googlescholar-id}{name-to-display}
%% \firstname and \lastname will be used
% \googlescholar{googlescholar-id}{}
% \extrainfo{extra informations}

% \quote{``Be the change that you want to see in the world.''}


%-------------------------------------------------------------------------------
\begin{document}

% Print the header with above personal informations
% Give optional argument to change alignment(C: center, L: left, R: right)
\makecvheader[C]

% Print the footer with 3 arguments(<left>, <center>, <right>)
% Leave any of these blank if they are not needed
\makecvfooter
{\today}
{Ben Liu~~~~~·~~~~~Resume}
{\thepage}


%-------------------------------------------------------------------------------
%	CV/RESUME CONTENT
%	Each section is imported separately, open each file in turn to modify content
%-------------------------------------------------------------------------------
%-------------------------------------------------------------------------------
%	SECTION TITLE
%-------------------------------------------------------------------------------
\cvsection{Summary}


%-------------------------------------------------------------------------------
%	CONTENT
%-------------------------------------------------------------------------------
\begin{cvparagraph}

  %---------------------------------------------------------
  A Senior Software Engineer with 10+ years of experience specializing in JavaScript ecosystems like ReactJS, NextJS, and JS runtimes. Passionate about open-source projects and skilled in leveraging tools like ChatGPT, SSG, and SSR to optimize SEO and improve search rankings. Proven ability to design scalable solutions, automate processes, and enhance user experience. Fluent in Mandarin and English, with expertise across multiple programming languages.
\end{cvparagraph}

%-------------------------------------------------------------------------------
%	SECTION TITLE
%-------------------------------------------------------------------------------
\cvsection{Work Experience}


%-------------------------------------------------------------------------------
%	CONTENT
%-------------------------------------------------------------------------------
\begin{cventries}

  %---------------------------------------------------------
  \cventry
  {Senior Software Engineer} % Job title
  {Agoda Company Pte. Ltd} % Organization
  {Bangkok, Thailand} % Location
  {Sep. 2019 - Present} % Date(s)
  {
    \begin{cvitems} % Description(s) of tasks/responsibilities
      % Activity Marketing Tech
      % SEO
      \item {Leading an SEO project, launched the MVP for 5K pages using SSG and SSR on React/.NET, achieving a 10\%+ ranking increase after two quarters.}
      \item {Built a pipeline to automate SEO feature processing and content optimization with GPT, reducing manual operations from 1 quarter to 1 week.}
      % Activity Booking
      \item {Organized a 30-person hackathon in 2023 to boost developer culture, resulting in 3 experiments that improved team experience.}
      % GTTD
      % \item {Built a pipeline for promoting activity products in Google Ads, integrating data querying, API, and transformation, boosting bookings by 5\%.}
      % Badges
      \item {Led BE design and implementation of a server-driven badge system with campaign and highlight badges, resulting in a 2\% booking increase.}
      % X-SELL
      \item {Led the FE POC for a cross-sell feature on hotel pages, collaborating with stakeholders and designers, resulting in a 3\% booking increase.}
      % MPMMB
      \item {Led front-end development of the “Manage My Booking” page in 1.5 quarters, facilitating customer management of activity bookings.}
        % \begin{itemize}
        %   \item {Collaborated with two teams to align project goals, risks, and updates, ensuring smooth progress.}
        %   \item {Refactored legacy dependencies, improving scalability and enabling a new flight product page launch within a quarter.}
        %   \item {Documented decision trees and design components, enhancing onboarding and reducing long-term maintenance efforts.}
        % \end{itemize}
      % CMS backoffice 2.0
      \item {Designed and developed an auto-content-replication process for UAT and prod, improving deployment efficiency and data consistency.}
      \item {Led a team to build a CMS backoffice, streamlining content validation and updates, detecting brand leaks, and improving workflow efficiency.}
      % Strategy Partner
      \item {Reduced partner websites' full-site launching time from a quarter to a month by leveraging branding services on 5 web and 7+ back-end services.}
      % Groupbuy Campaign
      % Mini Program
      \item {Achieved 13K+ bookings in China market within 1 day by building a scalable Serverless campaign API service to throttle peak traffic.}
      % \item {Grew 20\% month-on-month new users from China market by building a web client widget (WeChat-mini-program) in WeChat.}
      % \item {Increased 10\% of the team's fortnightly average task burning rate by sharing the react component behavioural test pattern and pairing in practice.}
      % \item {Reduced deploy time of 10-days-campaign page from 2 weeks to 1 day by creating a page generator that consumes data from CSV files.}
      % Campaign system
      % \item {Reduced deploy time of 1-day-campaign page from 3 days to half-day by designing and implementing the dynamic content system.}
      % \item {Resolved circular dependency on 90 files by analysing dependency diagrams and refactoring with extracting superclasses and interfaces.}
      % \item {Stopped growing HTTP request wrapper to 500+ lines of code from 100 when supporting GraphQL by refactoring with request interceptors.}
      % Member based coupon with game system
      % \item {Add the lottery game system in holiday events to attract new user to register from China, Hong Kong and Taiwan.}
    \end{cvitems}
  }

  %---------------------------------------------------------
  \cventry
  {Software Engineer} % Job title
  {Predictive Hire} % Organization
  {Melbourne, Australia} % Location
  {Oct. 2018 - Jun. 2019} % Date(s)
  {
    \begin{cvitems} % Description(s) of tasks/responsibilities
      % Platform + Predictor Py + CX
      % \item {Cut CI/CD run time by 600\% through microservices integration decoupling, enabling faster delivery cycles and better resource use.}
      % Page Up integration - ALDI
      % \item {Led team of 4 devs and 1 DevOps in migrating API services to microservices, reducing delivery time and boosting system resilience.}
      \item {Reduced deploy time from 4 hours to 30 mins by decoupling client-side from monolithic structure, improving scalability and efficiency.}
      % Reporting
      % \item {Guided 2 data specialists to process and deliver 1.5 GB monthly, ensuring accurate and timely customer reporting.}
    \end{cvitems}
  }

  %---------------------------------------------------------
  \cventry
  {Developer} % Job title
  {Tabcorp Holdings Limited} % Organization
  {Melbourne, Australia}
  {Jan. 2017 - Sep. 2018} % Date(s)
  {
    \begin{cvitems} % Description(s) of tasks/responsibilities
      % Transaction Geo Location recording service - integrate with Brisbane Tatt team
      \item {Built a scalable API handling 5K+ requests/min with the master-worker pattern, enhancing capacity and stability under heavy loads.}
      % Rubix integrate operator feedback system with Salesforce SSO
      % \item {Solved critical SSO API issues in production within a day by coordinating with Ops and Devs for smooth user authentication.}
      % Replace chorme headless with JSDOM for SVG unit tests
      % \item {Cut SVG test time by 3,500\% by switching from Chrome headless to JSDOM, accelerating test feedback.}
      % Rubix
      % \item {Boosted unit test coverage from 75\% to over 95\% by adding 100+ cases, ensuring higher code quality and reliability.}
      % Rubix performance issue - sync function call -> async function call
      \item {Throttled 150+ MB data streams to prevent browser crashes, enhancing system resilience for large data handling.}
      % Rubix performance issue - queueing the transaction before send them to Ag Table
      % \item {Improved transaction throughput by 500% through data queuing, reducing bottlenecks and speeding up processing.}
    \end{cvitems}
  }

  %---------------------------------------------------------
  \cventry
  {Software Engineer} % Job title
  {Hope Bay Technologies, Inc.} % Organization
  {Taipei, Taiwan} % Location
  {Feb. 2014 - Oct. 2016} % Date(s)
  {
    \begin{cvitems} % Description(s) of tasks/responsibilities
      % Human resource web app with google scripts and google docs
      \item {Replaced HR paperwork for 100+ staff with an internal tool built on Google Apps, reducing errors and streamlining operations.}
      % CC front end, CC back end, Monioring front end, Tera OS, Tera Storage
      % \item {Established 5 CI pipelines and 2 cross-compiling build systems for web apps, creating a robust integration process for development needs.}
    \end{cvitems}
  }

  %---------------------------------------------------------
\end{cventries}

%-------------------------------------------------------------------------------
%	SECTION TITLE
%-------------------------------------------------------------------------------
\cvsection{Projects}


%-------------------------------------------------------------------------------
%	CONTENT
%-------------------------------------------------------------------------------
\begin{cventries}

  %---------------------------------------------------------
  \cventry
  {Contributor} % Affiliation/role
  {\href{https://github.com/ocftw/open-star-ter-village}{\faGithubSquare\acvHeaderIconSep Open Star Ter Village}} % Project name
  {Open Culture Fundation Taiwan} % Group/Cooperators
  {Jul. 2021 - Present} % Date(s)
  {
    \begin{cvitems} % Description(s) of experience/contributions/knowledge
      \item {3 out of 50 youths joined open source projects after playing this open source project simulation board game on Google Apps and hard copies.}
    \end{cvitems}
  }

  %---------------------------------------------------------
  \cventry
  {Project Owner} % Affiliation/role
  {\href{https://github.com/fuboteam/weddi-app}{\faGithubSquare\acvHeaderIconSep Weddi App}} % Project name
  {FuboTeam} % Location
  {May 2018 - Present} % Date(s)
  {
    \begin{cvitems} % Description(s) of experience/contributions/knowledge
      \item {Sent realtime greetings to bride and groom in 2 wedding ceremonies on a ReactJS + firebase website created by 2 software engineers in a week.}
    \end{cvitems}
  }

  %---------------------------------------------------------
\end{cventries}

%-------------------------------------------------------------------------------
%	SECTION TITLE
%-------------------------------------------------------------------------------
\cvsection{Skills}


%-------------------------------------------------------------------------------
%	CONTENT
%-------------------------------------------------------------------------------
\begin{cvskills}

  %---------------------------------------------------------
  \cvskill
  {Front-end}
  {React, Redux, NextJS, Playwright, Puppeteer, Webpack, Babel}

  %---------------------------------------------------------
  \cvskill
  {Back-end}
  {.NET Core, Koa, GraphQL, RESTful API, SQL, NoSQL}

  %---------------------------------------------------------
  \cvskill
  {DevOps}
  {Docker, CI/CD (GitLab CI, GitHub actions), Dev Container}

  %---------------------------------------------------------
  \cvskill
  {Programming}
  {TypeScript/JavaScript (expert), python (proficient), C\# (proficient), scala, kotlin (working knowledge)}

  %---------------------------------------------------------
  % \cvskill
  % {Languages}
  % {Mandarin (native speaker), English (fluent)}

  %---------------------------------------------------------
\end{cvskills}

%-------------------------------------------------------------------------------
%	SECTION TITLE
%-------------------------------------------------------------------------------
\cvsection{Education}


%-------------------------------------------------------------------------------
%	CONTENT
%-------------------------------------------------------------------------------
\begin{cventries}

  %---------------------------------------------------------
  \cventry
  {Bachelor of Science in Mathematic} % Degree
  {NTU (National Taiwan University)} % Institution
  {Taipei, Taiwan} % Location
  {Jul. 2009 - Feb. 2014} % Date(s)
  {
    % \begin{cvitems} % Description(s) bullet points
    %   \item {Designed and practiced block chiper variants base on AES and Serpant}
    % \end{cvitems}
  }

  %---------------------------------------------------------
\end{cventries}



%-------------------------------------------------------------------------------
\end{document}
